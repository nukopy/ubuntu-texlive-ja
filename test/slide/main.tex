% This document is based on http://math.shinshu-u.ac.jp/~hanaki/beamer/beamer.html
\documentclass[dvipdfmx,cjk]{beamer}
%\documentclass[dvipdfm,cjk]{beamer} % オプションは環境や利用するプログラムによって変える
%\documentclass[dvips,cjk]{beamer}

% しおり(PDF にしたときの目次)の文字化け防止
\AtBeginDvi{\special{pdf:tounicode 90ms-RKSJ-UCS2}}
%\AtBeginDvi{\special{pdf:tounicode EUC-UCS2}}

% 右下のアイコンを消す
\setbeamertemplate{navigation symbols}{}

% テーマ
\usetheme{CambridgeUS}
%\usetheme{Boadilla}           %% Beamer のディレクトリの中の
%\usetheme{Madrid}             %% beamerthemeCambridgeUS.sty を指定
%\usetheme{Antibes}            %% 色々と試してみるといいだろう
%\usetheme{Montpellier}        %% サンプルが beamer\doc に色々とある。
%\usetheme{Berkeley}
%\usetheme{Goettingen}
%\usetheme{Singapore}
%\usetheme{Szeged}

%\usecolortheme{rose}          %% colortheme を選ぶと色使いが変わる
%\usecolortheme{albatross}

%\useoutertheme{shadow}                 %% 箱に影をつける
%\usefonttheme{professionalfonts}       %% 数式の文字を通常の LaTeX と同じにする

%\setbeamercovered{transparent}         %% 消えている文字をうっすらと表示する
\setbeamertemplate{theorems}[numbered]  %% 定理に番号をつける
\newtheorem{thm}{Theorem}[section]
\newtheorem{proposition}[thm]{Proposition}
\theoremstyle{example}
\newtheorem{exam}[thm]{Example}
\newtheorem{remark}[thm]{Remark}
\newtheorem{question}[thm]{Question}
\newtheorem{prob}[thm]{Problem}

% メタ情報
\begin{document}
\title[Sample of Beamer]{Beamer サンプル}
\author[nukopy]{nukopy}
\institute[nukopy org.]{nukopy の所属}
\date{February 6, 2007}

% タイトルスライド
\begin{frame}
\titlepage
\end{frame}

% 目次(\section 名が自動で挿入される)
\begin{frame}
\tableofcontents
\end{frame}

% セクション名(サンプルでは左上のスペースに表示される)
\section{箇条書き}
\begin{frame}
\frametitle{プログラミング言語} % スライドのタイトル

\begin{itemize}
\item TypeScript\pause % \pause でスライドの分割できる
\item Go\pause
\item LaTeX
\end{itemize}
\end{frame}

% 定理
\section{定理型環境}
\begin{frame} % \newtheorem で新しい環境も作れる
\begin{thm}
定理型環境が使える。
使い方は普通の \LaTeX と同じ
\end{thm}
\pause

% 証明
\begin{proof}
証明も書ける。
\end{proof}
\pause

% 例
\begin{exam}
example
\end{exam}
\end{frame}

% 文字色を変える
\section{文字の色を変える}             %% 文字の色を変える
\begin{frame}
\frametitle{文字の色を変えてみよう!}
{\color{red}赤}\pause
{\color{blue}青}\pause
{\color{green}緑}
\end{frame}

\end{document}
